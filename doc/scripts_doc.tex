\documentclass[a4paper,10pt,oneside]{memoir}
\usepackage[polutonikogreek,italian]{babel}
\usepackage[utf8x]{inputenc}
\usepackage{amsmath}
\usepackage{amsthm}
\usepackage{amssymb}
\usepackage{amscd}
\usepackage{graphicx}
\usepackage{float}
\usepackage{array}
\usepackage{rotating}
\usepackage[small]{caption}
\usepackage{lscape}
\usepackage{fancybox}
\usepackage{booktabs}
\usepackage[noanswer]{exercise}
\usepackage{color,calc,graphicx,soul,fourier}
\definecolor{nicered}{rgb}{.647,.129,.149}
\makeatletter
\newlength\dlf@normtxtw
\setlength\dlf@normtxtw{\textwidth}
\def\myhelvetfont{\def\sfdefault{mdput}}
\newsavebox{\feline@chapter}
\newcommand\feline@chapter@marker[1][4cm]{%
  \sbox\feline@chapter{%
    \resizebox{!}{#1}{\fboxsep=1pt%
      \colorbox{nicered}{\color{white}\bfseries\sffamily\thechapter}%
    }}%
  \rotatebox{90}{%
    \resizebox{%
      \heightof{\usebox{\feline@chapter}}+\depthof{\usebox{\feline@chapter}}}%
    {!}{\scshape\so\@chapapp}}\quad%
  \raisebox{\depthof{\usebox{\feline@chapter}}}{\usebox{\feline@chapter}}%
}
\newcommand\feline@chm[1][4cm]{%
  \sbox\feline@chapter{\feline@chapter@marker[#1]}%
  \makebox[0pt][l]{% aka \rlap
    \makebox[1cm][r]{\usebox\feline@chapter}%
  }}
\makechapterstyle{daleif1}{
  \renewcommand\chapnamefont{\normalfont\Large\scshape\raggedleft\so}
  \renewcommand\chaptitlefont{\normalfont\huge\bfseries\scshape\color{nicered}}
  \renewcommand\chapternamenum{}
  \renewcommand\printchaptername{}
  \renewcommand\printchapternum{\null\hfill\feline@chm[2.5cm]\par}
  \renewcommand\afterchapternum{\par\vskip\midchapskip}
  \renewcommand\printchaptertitle[1]{\chaptitlefont\raggedleft ##1\par}
}
\makeatother
\chapterstyle{daleif1}

\setlength\afterchapskip {\onelineskip }
\setlength\beforechapskip {\onelineskip }




\usepackage{hyperref}
\renewcommand{\textfraction}{0.05}
\renewcommand{\topfraction}{0.95}
\renewcommand{\bottomfraction}{0.95}
\renewcommand{\floatpagefraction}{0.35}
\renewcommand{\ExerciseName}{Esercizio}
\renewcommand{\ExerciseListName}{Es}
\setcounter{totalnumber}{5}
\restylefloat{figure}
\begin{document}
 \chapter{Script gestione server}
    \section*{cleanup}
      Lo script \verb#cleanup# si occupa della rimozione di dati indesiderati, come profili (samba) di utenti non più presenti, cestini di rete troppo pieni, file indesiderati (video, immagini, etc). Lo script può essere invocato come:
      \begin{verbatim}
       cleanup [options] --item [item]
      \end{verbatim}
      \begin{itemize}
	\item \verb#--item#, gli elementi implementati sono, per ora, \verb#dustbins,profiles#  
      \end{itemize}
\section*{archive\_mail}

Questo script si occupa dell'archiviazione delle mail ricevuto dal protocollo, è in grado di processare un file mbox, di estrarre i nuovi messaggi ricevuti e iniettarli nel server di gestione mail \emph{Archiveopteryx}. Lo script può essere invocato come:
\begin{verbatim}
 archive_mail --folder [folder]
\end{verbatim}
questo script viene eseguito giornalmente via cron
\section*{genlogon}
Questo script viene eseguito da \emph{samba} durante il logon degli utenti su macchine Windows, tale script genera automaticamente uno script vbs che verrà eseguito dal client. I parametri per la creazione dello script sono specificati nel file di configurazione \verb#logon_prefs.yaml#


 \chapter{Script gestione client}
\section*{list\_clients}
  Lo script \verb#list_clients# mostra i client registrati sul server corrente e controlla se questi sono attivi. Lo script può essere invocato come:
\begin{verbatim}
 list_clients --group [group]
\end{verbatim}
Il parametro \verb#--group# è opzionale, può essere utilizzato per visualizzare lo stato di un gruppo di computer, ad esempio è possibile controllare i pc di un laboratorio:
\begin{verbatim}
 list_clients --group lab1b
\end{verbatim}
\section*{run}
Lo script \verb#run# permette l'esecuzione di comandi predefiniti nel file di configurazione \verb#commands.yaml#. Lo script può essere invocato come:
\begin{verbatim}
 run --client [client] --command [command] [--group [group]] [--all] [--list-commands]
\end{verbatim}
gli switch \verb#--client# e \verb#--group# sono mutamente esclusivi. Lo switch \verb#--list-commands# stampa una lista dei comandi eseguibili sui client
\section*{shutdown\_client}
Lo script \verb#shutdown_client# si occupa dello spegnimento dei client Windows. Lo script può essere invocato come:

\begin{verbatim}
 shutdown_client --client [client] [--group [group]] [--all]]
\end{verbatim}
come negli altri casi gli switch \verb#--client# e \verb#--group# sono mutamente esclusivi. Nella versione attuale dello script è necessario specificare lo switch \verb#--all# quando si fa uso dello switch \verb#--group#

\section*{update\_clients}

Lo script \verb#udpate_clients# nella sua versione attuale deve essere utilizzato per l'aggiornamento simultaneo di un gruppo di computer. Per l'aggiornamento di una singolo macchina si dovrà invece utilizzare lo script \verb#update_client# in un futura revisione degli script queste due funzionalità saranno integrate in un unico comando. Lo script può essere invocato come:
\begin{verbatim}
 update_clients --group [group]
\end{verbatim}
se lo switch \verb#--group# non viene specificato tutti i client registrati sul server verranno aggiornati.

\section*{wakeup\_client}

Lo script \verb#wakeup_client# permette l'accensione remota di tutti quei computer il cui hardware supporta il \emph{Wake on Lan}. Lo script può essere invocato come:
\begin{verbatim}
 wakup_client --client [client] [--group [group]] [--all]
\end{verbatim}
Gli switch \verb#--client# e \verb#--group# sono mutamente esclusivi inoltre nel caso in cui venga specificato un gruppo è necessario fornire anche lo switch \verb#--all#

\section*{wpkg\_report}

Lo script \verb#wpkg_report# estrae dalle macchine client le informazioni sull'aggiornamento dei pacchetti software, eseguito tramite lo script \emph{wpkg}. Lo script può essere invocato come:
\begin{verbatim}
 wpkg_report --client [client] [--group [group]] [--all] [--reportonly]
\end{verbatim}
Gli switch \verb#--client# e \verb#--group# sono mutamente esclusivi, in caso di selezione di un gruppo è necessario specificare lo switch \verb#--all#. Nel caso in cui non si voglia scaricare dai client le informazioni sui pacchetti (ad esempio nel caso in cui queste siano già state acquisite) è possibile utilizzare lo switch \verb#--reportonly# per effettuare unicamente il report degli esiti di installazione.

\end{document}
